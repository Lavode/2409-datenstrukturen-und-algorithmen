\documentclass[a4paper]{scrartcl}
%\documentclass[a4paper]{report}

% Uncomment to optimize for double-sided printing.
% \KOMAoptions{twoside}

% Set binding correction manually, if known.
% \KOMAoptions{BCOR=2cm}

% Localization options
\usepackage[english]{babel}
\usepackage[T1]{fontenc}
\usepackage[utf8]{inputenc}

% Enhanced verbatim sections. We're mainly interested in
% \verbatiminput though.
\usepackage{verbatim}

% PDF-compatible landscape mode.
% Makes PDF viewers show the page rotated by 90°.
\usepackage{pdflscape}

% Advanced tables
\usepackage{tabu}
\usepackage{longtable}

% Fancy tablerules
\usepackage{booktabs}

% Graphics
\usepackage{graphicx}

% Current time
\usepackage[useregional=numeric]{datetime2}

% Float barriers.
% Automatically add a FloatBarrier to each \section
\usepackage[section]{placeins}

% Custom header and footer
% \usepackage{fancyhdr}
% \setlength{\headheight}{15.2pt}
% \pagestyle{fancyplain}

\usepackage{geometry}
\usepackage{layout}

\usepackage{subcaption}

% Math tools
\usepackage{mathtools}
% Math symbols
\usepackage{amsmath,amsfonts,amssymb}

% \fancyhf{}
% % Chapter header on non-plain pages only.
% \lhead{\fancyplain{} {\leftmark}}
% % Footer must contain print date. Ugly, but IPA requirement.
% \lfoot{\printdate}
% % Print date left and page count right was the thing which looked the
% % most balanced.
% \rfoot{\thepage}
% 
% Source code & highlighting
\usepackage{listings}

% Convenience commands
\newcommand{\mailsubject}{2409 - Datenstrukturen und Algorithmen - Series 4}
\newcommand{\maillink}[1]{\href{mailto:#1?subject=\mailsubject}
                               {#1}}

% Should use this command wherever the print date is mentioned.
\newcommand{\printdate}{\today}

\subject{2409 - Datenstrukturen und Algorithmen}
\title{Series 4, Theoretical Exercises}

\author{Michael Senn - 16-126-880}

\date{}

% Needs to be the last command in the preamble, for one reason or
% another. 
\usepackage{hyperref}


\begin{document}
\maketitle

% \tableofcontents

\section{Illustrating Counting-Sort}

\begin{verbatim}
	A = [2, 7, 1, 3, 2, 4, 1, 8, 5, 1, 4]
	C = [3, 2, 1, 2, 1, 0, 1, 1]

	=>

	C = [3, 5, 6, 8, 9, 9, 10, 11]

	=>

	B = [, , , , , , , 4, , , ]
	C = [3, 5, 6, 7, 9, 9, 10, 11]

	=>

	B = [, , 1, , , , , 4, , , ]
	C = [2, 5, 6, 7, 9, 9, 10, 11]

	=>

	B = [, , 1, , , , , 4, 5, , ]
	C = [2, 5, 6, 7, 8, 9, 10, 11]

	=>

	B = [, , 1, , , , , 4, 5, , 8]
	C = [2, 5, 6, 7, 8, 9, 10, 10]

	=>

	B = [, , 1, , , , 4, 4, 5, , 8]
	C = [2, 5, 6, 6, 8, 9, 10, 10]

	=>

	B = [, 1, 1, , , , , 4, 5, , 8]
	C = [1, 5, 6, 7, 8, 9, 10, 10]

	=>

	B = [, 1, 1, , , , 4, 4, 5, , 8]
	C = [1, 5, 6, 6, 8, 9, 10, 10]

	=>

	B = [, 1, 1, , 2, , 4, 4, 5, , 8]
	C = [1, 4, 6, 6, 8, 9, 10, 10]

	=>

	B = [, 1, 1, , 2, 3, 4, 4, 5, , 8]
	C = [1, 4, 5, 6, 8, 9, 10, 10]

	=>

	B = [1, 1, 1, , 2, 3, 4, 4, 5, , 8]
	C = [0, 4, 5, 6, 8, 9, 10, 10]

	=>

	B = [1, 1, 1, , 2, 3, 4, 4, 5, 7, 8]
	C = [0, 4, 5, 6, 8, 9, 9, 10]

	=>

	B = [1, 1, 1, 2, 2, 3, 4, 4, 5, 7, 8]
	C = [0, 3, 5, 6, 8, 9, 9, 10]
\end{verbatim}

\section{Number of items in interval}

Let \texttt{input} = $[x_1, ..., x_n], x_i \in \{1, ..., k\}$

We start by preparing two arrays which will be used to determine the number of
items in a given interval in a later step.

\begin{verbatim}
	occurences = []
	less_than  = []

	for i = 1 to k:
	  occurences[i] = 0
	  less_than[i] = 0

	for i = 1 to n:
	  occurences[input[i]] += 1

	for i = 2 to k:
	  less_than[i] = less_than[i - 1] + occurences[i - 1]
\end{verbatim}

Clearly, the first and last loops have a time complexity of $\theta(k)$, while
the middle loop has a complexity of $\theta(n)$. As such, the preparation step
has a time complexity of $\theta(k + n)$.

Now we can calculate the number of items in the interval $[a, b]$ as:
\begin{verbatim}
	less_than[b] - less_than[a] + occurences[b]
\end{verbatim}

\section{Illustrating Radix-sort}

\begin{verbatim}
    CAT        TE*A*        M*A*P        *A*RT
    DOG        BE*D*        C*A*R        *B*ED
    CAR        DO*G*        C*A*T        *C*AR
    CUP        PI*G*        R*A*T        *C*AT
    PIG        FO*G*        D*A*Y        *C*UP
    MAP        NI*L*        T*E*A        *D*AY
    DAY        CU*P*        B*E*D        *D*OG
    ART   =>   MA*P*   =>   P*I*G   =>   *F*OG
    BED        CA*R*        N*I*L        *G*UY
    FOG        CA*T*        D*O*G        *M*AP
    GUY        AR*T*        F*O*G        *N*IL
    NIL        RA*T*        A*R*T        *P*IG
    RAT        DA*Y*        C*U*P        *R*AT
    TEA        GU*Y*        G*U*Y        *T*EA
\end{verbatim}

\section{Stability of sorting algorithms}

Implementations of insertion sort will usually be stable, since they will sort
left-to-right, moving each to-be-sorted item right-to-left until the first spot
it can fit into.
\\ 
Implementations of merge sort will usually be stable, due to - during the
merge phase - iterating over all blocks, as well as over all items within each
block - in one direction.
\\
Heapsort will not be stable, as the initial stage - where the heap is created -
will not preserve relative ordering of elements. That is, a heap provides
absolute order, but does not preserve relative one.
\\
Quicksort will not be stable, as each phase only ensures that the position of
elements relative to the pivot is correct - but makes no guarantee about the
position of elements relative to each other.

\subsection{Making unstable algorithms stable}

Comparison-based unstable sorting algorithms can be easily made stable - at the
cost of additional memory and processing time - by extending the sorting key by
the position of the item in the original input list. That is, to use the item's
original position as a 'tie-breaker' in case of items which are equal based on
their sorting key.
\end{document}
