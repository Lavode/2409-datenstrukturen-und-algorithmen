\documentclass[a4paper]{scrartcl}
%\documentclass[a4paper]{report}

% Uncomment to optimize for double-sided printing.
% \KOMAoptions{twoside}

% Set binding correction manually, if known.
% \KOMAoptions{BCOR=2cm}

% Localization options
\usepackage[english]{babel}
\usepackage[T1]{fontenc}
\usepackage[utf8]{inputenc}

% Enhanced verbatim sections. We're mainly interested in
% \verbatiminput though.
\usepackage{verbatim}

% PDF-compatible landscape mode.
% Makes PDF viewers show the page rotated by 90°.
\usepackage{pdflscape}

% Advanced tables
\usepackage{tabu}
\usepackage{longtable}

% Fancy tablerules
\usepackage{booktabs}

% Graphics
\usepackage{graphicx}

% Current time
\usepackage[useregional=numeric]{datetime2}

% Float barriers.
% Automatically add a FloatBarrier to each \section
\usepackage[section]{placeins}

% Custom header and footer
% \usepackage{fancyhdr}
% \setlength{\headheight}{15.2pt}
% \pagestyle{fancyplain}

\usepackage{geometry}
\usepackage{layout}

\usepackage{subcaption}

% Math tools
\usepackage{mathtools}
% Math symbols
\usepackage{amsmath,amsfonts,amssymb}

% \fancyhf{}
% % Chapter header on non-plain pages only.
% \lhead{\fancyplain{} {\leftmark}}
% % Footer must contain print date. Ugly, but IPA requirement.
% \lfoot{\printdate}
% % Print date left and page count right was the thing which looked the
% % most balanced.
% \rfoot{\thepage}
% 
% Source code & highlighting
\usepackage{listings}

% Convenience commands
\newcommand{\mailsubject}{2409 - Datenstrukturen und Algorithmen - Series 5}
\newcommand{\maillink}[1]{\href{mailto:#1?subject=\mailsubject}
                               {#1}}

% Should use this command wherever the print date is mentioned.
\newcommand{\printdate}{\today}

\subject{2409 - Datenstrukturen und Algorithmen}
\title{Series 5}

\author{Michael Senn - 16-126-880}

\date{}

% Needs to be the last command in the preamble, for one reason or
% another. 
\usepackage{hyperref}


\begin{document}
\maketitle

\section{Reversing a linked list}

The following method - shown in the (fictious) environment of a \texttt{LinkedList}
class - will reverse the list.

\begin{lstlisting}[language=ruby]
class Node
    def next
        # ...
    end

    def next=(val)
        # ...
    end
end
class LinkedList
    def reverse
        prev_node = nil 
        cur_node  = @head

        while next_node = cur_node.next
            cur_node.next       = prev_node
            prev_node, cur_node = cur_node , next_node
        end 

        cur_node.next = prev_node
        @head         = cur_node
    end
end
\end{lstlisting}


% \tableofcontents

\section{Implementing a queue with a linked list}

Below you will find the relevant parts of an implementation of a queue using
linked lists, with support for constant-time enqueueing and dequeueing.

In addition you will find visualizations of it during operation.

As an implementation detail the queue will - when calling \texttt{dequeue} on
an empty queue - return \texttt{nil}, rather than throwing an exception.

\begin{lstlisting}[language=ruby]
class Queue
    def enqueue(val)
        node = Node.new(val)

        @tail.next = node if @tail
        @tail = node
        @head = node unless @head
    end

    def dequeue
        return nil unless @head

        ret = @head
        @head = ret.next
        ret.next = nil

        return ret
    end
end
\end{lstlisting}

\begin{lstlisting}
Enqueueing 3
Queue is now: [[HEAD]<3->nil>[TAIL]]

Enqueueing 5
Queue is now: [[HEAD]<3->5>, <5->nil>[TAIL]]

Dequeued: 3
Queue is now: [[HEAD]<5->nil>[TAIL]]

Enqueueing 2
Queue is now: [[HEAD]<5->2>, <2->nil>[TAIL]]

Dequeued: 5
Queue is now: [[HEAD]<2->nil>[TAIL]]

Enqueueing 8
Queue is now: [[HEAD]<2->8>, <8->nil>[TAIL]]

Enqueueing 9
Queue is now: [[HEAD]<2->8>, <8->9>, <9->nil>[TAIL]]

Dequeued: 2
Queue is now: [[HEAD]<8->9>, <9->nil>[TAIL]]

Dequeued: 8
Queue is now: [[HEAD]<9->nil>[TAIL]]

Dequeued: 9
Queue is now: []

Dequeued: nil
Queue is now: []
\end{lstlisting}

\section{Depth-first traversal of directed LCRS tree}

Following we define a method which allows to do recursive depth-first traversal
of an LCRS-style directed tree. The method will traverse all nodes below the
given node, including its siblings.  As such, in order to traverse the whole
tree, the root node has to be passed.

Finding the root node would have to be done outside of the traversing function,
to prevent infinite loops. As this would be trivial - one can simply follow the
pointers to each node's parents to the very top - it was left out.

\begin{lstlisting}[language=ruby]
def traverse(node)
    puts node.key

    traverse(node.left_child)    if node.left_child
    traverse(node.right_sibling) if node.right_sibling
end
\end{lstlisting}

\section{Non-recursive traversal of directed LCRS tree}

Following we define a method which allows to non-recursively traverse an
LCRS-style directed tree. It uses a stack to keep track of discovered, but
not-yet-traversed, nodes, to come back to later if the current path was
exhausted.

Similarly to above, the root node has to be passed in in order to traverse the
full tree.

\begin{lstlisting}[language=ruby]
def traverse(node)
    stack = Stack.new
    stack.push(node)

    while (node = stack.pop) do
      puts node.key

      stack.push(node.left_child)    if node.left_child
      stack.push(node.right_sibling) if node.right_sibling
    end
end
\end{lstlisting}

\end{document}
