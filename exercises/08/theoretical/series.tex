\documentclass[a4paper]{scrartcl}
%\documentclass[a4paper]{report}

% Uncomment to optimize for double-sided printing.
% \KOMAoptions{twoside}

% Set binding correction manually, if known.
% \KOMAoptions{BCOR=2cm}

% Localization options
\usepackage[english]{babel}
\usepackage[T1]{fontenc}
\usepackage[utf8]{inputenc}

% Enhanced verbatim sections. We're mainly interested in
% \verbatiminput though.
\usepackage{verbatim}

% PDF-compatible landscape mode.
% Makes PDF viewers show the page rotated by 90°.
\usepackage{pdflscape}

% Advanced tables
\usepackage{tabu}
\usepackage{longtable}

% Fancy tablerules
\usepackage{booktabs}

% Graphics
\usepackage{graphicx}

% Current time
\usepackage[useregional=numeric]{datetime2}

% Float barriers.
% Automatically add a FloatBarrier to each \section
\usepackage[section]{placeins}

% Custom header and footer
% \usepackage{fancyhdr}
% \setlength{\headheight}{15.2pt}
% \pagestyle{fancyplain}

\usepackage{geometry}
\usepackage{layout}

\usepackage{subcaption}

% Math tools
\usepackage{mathtools}
% Math symbols
\usepackage{amsmath,amsfonts,amssymb}

% \fancyhf{}
% % Chapter header on non-plain pages only.
% \lhead{\fancyplain{} {\leftmark}}
% % Footer must contain print date. Ugly, but IPA requirement.
% \lfoot{\printdate}
% % Print date left and page count right was the thing which looked the
% % most balanced.
% \rfoot{\thepage}
% 
% Source code & highlighting
\usepackage{listings}

\DeclarePairedDelimiter\floor{\lfloor}{\rfloor}
\DeclarePairedDelimiter\abs{\lvert}{\rvert}

% Convenience commands
\newcommand{\mailsubject}{2409 - Datenstrukturen und Algorithmen - Series 8}
\newcommand{\maillink}[1]{\href{mailto:#1?subject=\mailsubject}
                               {#1}}

% Should use this command wherever the print date is mentioned.
\newcommand{\printdate}{\today}

\subject{2409 - Datenstrukturen und Algorithmen}
\title{Series 8}

\author{Michael Senn - 16-126-880}

\date{}

% Needs to be the last command in the preamble, for one reason or
% another. 
\usepackage{hyperref}


\begin{document}
\maketitle

\section{Relaxed red-black tree}

Changing the previously-red root node of a relaxed red-black tree to black will
result in a regular red-black tree, with all requirements fulfilled.

This can be shown by looking at the five properties of a red-black tree.

\begin{itemize}
	\item Every node is either red or black: Not affected by change.
	\item Root node is black: Fulfilled by change.
	\item Every leaf is black: Not affected by change.
	\item Red nodes have black children: Not affected by change, as no requirement for black nodes.
	\item Black height of one node to all its children is equal: Not affected by change, as node itself not counted for black height.
\end{itemize}


\section{Path lenghts}

In order to show that the longest path from a node to any of its leaves is, at
most, twice as long as the shortest path from the same node to any of its
leaves, we look at how these paths are made up.

Due to property five, the black-height of both paths - meaning the number of
black nodes in each path - will be equal. As such, it is clear that the
shortest path will consist of solely black nodes, wheras the longest path will
contain as many red nodes as possible.

Due to property four, however, the longest path can not consist of more than
$\frac{1}{2}$ red nodes, as each red node has to be followed by a black one.

As such, the length of the shortest path will be $bh(n)$, and the length of the
longest path $bh(n) + rh(n)$, with $rh(n) \leq bh(n)$. As such, the longest
path will be at most twice as long as the shortest one.

\section{Inner nodes}

Following up from the previous exercise, a tree with minimum number of inner
nodes - that is nodes excluding leaf nodes - will consist of solely black
nodes. As such, such a tree will consist of $2^{bh(root)} - 1$ inner nodes.

Similarly, a tree with maximum number of inner nodes will contain of solely
longest paths, with equal amounts of red and black nodes, for a total of $2^{2
* bh(root)} - 1$ inner nodes.

\section{Right-Rotate}

Based on the left-rotate algorithm presented in the lecture.

\begin{lstlisting}
right_rotate(T, x):
  y = x.left
  x.left = y.right

  if y.right != T.nil:
    y.right.p = x
  y.p = x.p

  if x.p == T.nil:
    T.root = y
  elsif x == x.p.right:
    x.p.right = y
  else:
    x.p.right = y

  y.right = x
  x.p = y
\end{lstlisting}

\section{Transforming binary trees}

Each binary tree can be transformed into a right-leaning chain by applying a
right-rotation to all nodes with a left child. As each of these right-rotations
will increase the number of nodes in the right chain by one, there are at most
$n-1$ rotations needed to transform the tree into a right-leaning chain.

Afterwards, any number of left-rotations can be applied to create a new tree
structure, with the original tree resulting if as many left-rotations are
applied, as right-rotations were applied previously.

As such, any binary tree can be transformed into another by a sequence of
right-rotations, followed by a sequence of left-rotations, with an upper bound
of $O(n)$.

\section{Manual tree building}

Handed in on paper.

\section{Effect of transformations on black-length}

Handed in on paper.

\section{Effect of insertion and subsequent deletion on structure}

Handed in on paper.

\end{document}
