\documentclass[a4paper]{scrartcl}
%\documentclass[a4paper]{report}

% Uncomment to optimize for double-sided printing.
% \KOMAoptions{twoside}

% Set binding correction manually, if known.
% \KOMAoptions{BCOR=2cm}

% Localization options
\usepackage[english]{babel}
\usepackage[T1]{fontenc}
\usepackage[utf8]{inputenc}

% Enhanced verbatim sections. We're mainly interested in
% \verbatiminput though.
\usepackage{verbatim}

% PDF-compatible landscape mode.
% Makes PDF viewers show the page rotated by 90°.
\usepackage{pdflscape}

% Advanced tables
\usepackage{tabu}
\usepackage{longtable}

% Fancy tablerules
\usepackage{booktabs}

% Graphics
\usepackage{graphicx}

% Current time
\usepackage[useregional=numeric]{datetime2}

% Float barriers.
% Automatically add a FloatBarrier to each \section
\usepackage[section]{placeins}

% Custom header and footer
% \usepackage{fancyhdr}
% \setlength{\headheight}{15.2pt}
% \pagestyle{fancyplain}

\usepackage{geometry}
\usepackage{layout}

\usepackage{subcaption}

% Math tools
\usepackage{mathtools}
% Math symbols
\usepackage{amsmath,amsfonts,amssymb}

% \fancyhf{}
% % Chapter header on non-plain pages only.
% \lhead{\fancyplain{} {\leftmark}}
% % Footer must contain print date. Ugly, but IPA requirement.
% \lfoot{\printdate}
% % Print date left and page count right was the thing which looked the
% % most balanced.
% \rfoot{\thepage}
% 
% Source code & highlighting
\usepackage{listings}

% Allow breaking '0' and '1'
\lstset{literate=
	{-}{{-\allowbreak}}{1}
	{0}{{0\allowbreak}}{1}
	{1}{{1\allowbreak}}{1}
}

\DeclarePairedDelimiter\floor{\lfloor}{\rfloor}
\DeclarePairedDelimiter\abs{\lvert}{\rvert}

% Convenience commands
\newcommand{\mailsubject}{2409 - Datenstrukturen und Algorithmen - Series 10}
\newcommand{\maillink}[1]{\href{mailto:#1?subject=\mailsubject}
                               {#1}}

% Should use this command wherever the print date is mentioned.
\newcommand{\printdate}{\today}

\subject{2409 - Datenstrukturen und Algorithmen}
\title{Series 10}

\author{Michael Senn - 16-126-880}

\date{}

% Needs to be the last command in the preamble, for one reason or
% another. 
\usepackage{hyperref}


\begin{document}
\maketitle

\section{Greedy Algorithms - Activity planning}

Handed in on paper.

\section{Huffman Coding}

Handed in on paper.

\section{Change-making}

\subsection{Greedy algorithm}

We define the following naive algorithm

\begin{itemize}
	\item Let $A$ be empty array.
	\item Let $amount$  be amount of change which to hand out.
	\item While $amount \neq 0$
		\begin{itemize}
			\item Let $c$ be coin with biggest denomination so $c \leq amount$
			\item Append $c$ to $A$
			\item $amount := amount - c$
		\end{itemize}
	\item Return $A$
\end{itemize}

In words - starting with the biggest denomination, use as many of that coin as
possible, then move on to the next smaller denomination. Continue until
requested amount is met.

\subsubsection{Optimal substructure}

We intend to show that an optimal solution for a given problem also contains
optimal solutions for its subproblems.

Let $a$ be the amount for which to hand out change, let $f(a)$ be the number of coins
of the ideal solution for amount $a$.

Assume $x = f(a)$ for a given $a$. Further assume that there exists a coin $c$
in the optimal solution.

This implies that the given solution for amount $a$ also contains a solution -
with $x - 1$ coins - for amount $a - c$.

If this solution to the subproblem were not optimal, then there would exist a
solution - with $y < x - 1$ coins - for amount $a - c$. This would, however,
imply that there would also exist a solution with $z < x$ coins for amount $a$.

As this is a contradiction to the original assumption, we have shown that the
problem has an optimal substructure.

\subsubsection{Greedy choice property}

By differentiating the following four cases, we can show that - in any step of
the iteration - the coin picked by our algorithm will be part of the ideal
solution.

Let $amount$ be the amount of change which to hand out.

\begin{description}
	\item[$amount \geq 25$] The optimal solution is guaranteed to cointain at least one coin with denomination $25$.
	\item[$25 > amount \geq 10$] The optimal solution is guaranteed to contain 1 -- 2 coins with denomination $10$.
	\item[$10 > amount \geq 5$] The optimal solution is guaranteed to contain exactly one coin with denomination $5$.
	\item[$5 > amount \geq 1$] The optimal solution is guaranteed to contain 1 -- 4 coins with denomination $1$.
\end{description}

As such, the choice our algorithm makes is guaranteed to be part of the optimal
solution.

\subsection{Coin systems where greedy algorithm fails}

One example, where our greedy algorithm fails to produce the optimal solution
is the following:

Let our coin system cointain the following denominations: $N = (1, 3, 4, 5)$

When trying to find optimal change for $n = 7$ cents, the algorithm would come up
with $(5, 1, 1)$, whereas the optimal solution would be $(4, 3)$.

The same issue will arise with eg $N = (25, 10, 1)$, $n = 40$.

\subsection{Change-making via dynamic programming}

Let $N = (c_1, c_2, \dots, c_k)$ be our coins.

We define the list containing number of coins in the optimal solution as follows:
\[
	c(j) = 
	\begin{cases} 
		\infty & j < 0  \\
		0 & j = 0 \\
		\min_{x \in \{1, 2, \dots, k\}} 1 + c(j - c_x) & \text{else}
	\end{cases}
\]

At the same time we use $denom(j)$ to keep track of the denomination of a coin
used to create the ideal solution for amount $j$. This can be tracked easily
within implementations generating the table above, by setting $denom(j) = c_j$
for those cases where we set $c(j) = 1 + c(j - c_x)$, ie those cases where a
better solution is found when considering an additional coin.

These two lists can be created with a time-complexity of $\theta(n * k)$.

Having created those two lists, calculating the optimal solution for a given
amount $n$ is as easy as looking at $denom(n)$, adding said coin to our
solution, setting $n = n - denom(n)$, and repeating while $n > 0$.

Illustrating the above as a recursive function:
\begin{lstlisting}
	change(j, denom):
	  if j > 0:
	    return [denom[j]] + change(j - denom[j], denom)
\end{lstlisting}

\section{Practical exercise}

The results of encoding the two provided sentences using Huffman coding are attached below.

The number of bits per input character for these two sentences was at 4.01 and
4.08 respectively. Asuming ASCII (or UTF-8) encoding, this amounts to a
compression of nearly 50\% excluding the encoding table, which would
have to be transmitted in addition to the payload unless additional rules were
established to remove ambiguity.

\begin{lstlisting}[breaklines=true]
Input:
Jobs launched into a sermon about how the Macintosh and its software would be so easy to use that there would be no manuals.
Output:
01000111111100111010001000111000110011110010101011110101100010111101111101111100110000101011100100110111011110111001100100111111011011010001011111100000011010101111000100100111001001011011110111110111111010010100110001111011000101111110110100010101111010001011011000011000100111100010000111101101000110110001001111100010101111001110110010101001000001101111100011010101110001101010111001101001101010111100100111100010000111101101000110110001001111100001111111001011101100011101101100100011010010000
Input length [chars]: 124, Output length [bits]: 497, Bits per char: 4.01
-------
Input:
An academic career in which a person is forced to produce scientific writings in great amounts creates a danger of intellectual superficiality, Einstein said.
Output:
001000010101101110101111101001000000100101010111101011111001110000000111110010101011000101100101001010110010101101110110111100000011101101001110101100100110110111111100110111101100010010110100010011110111100011110011100100011010110001100110101101000010101000010111111010101111000101101110101000010101011111001101100101010110111110011100011101000110111000100110011001101010100001101101011011100011101000000011011011101101001011101010111110000011111010011111111110010101010000000011100111000101110000011011100011111001100011011110000001111111110101011010111000111010100011110101111011111011110110010101001101000000010101011001101110010100100010001
Input length [chars]: 158, Output length [bits]: 645, Bits per char: 4.08
\end{lstlisting}	

\end{document}
